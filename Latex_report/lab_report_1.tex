%%%%%%%%%%%%%%%%%%%%%%%%%%%%%%%%%%%%%%%%%
% University/School Laboratory Report
% LaTeX Template
% Version 3.1 (25/3/14)
%
% This template has been downloaded from:
% http://www.LaTeXTemplates.com
%
% Original author:
% Linux and Unix Users Group at Virginia Tech Wiki 
% (https://vtluug.org/wiki/Example_LaTeX_chem_lab_report)
%
% License:
% CC BY-NC-SA 3.0 (http://creativecommons.org/licenses/by-nc-sa/3.0/)
%
%%%%%%%%%%%%%%%%%%%%%%%%%%%%%%%%%%%%%%%%%

%----------------------------------------------------------------------------------------
%	PACKAGES AND DOCUMENT CONFIGURATIONS
%----------------------------------------------------------------------------------------

\documentclass{article}

\usepackage[version=3]{mhchem} % Package for chemical equation typesetting
\usepackage{siunitx} % Provides the \SI{}{} and \si{} command for typesetting SI units
\usepackage{graphicx} % Required for the inclusion of images
\usepackage{natbib} % Required to change bibliography style to APA
\usepackage{amsmath} % Required for some math elements 

\setlength\parindent{0pt} % Removes all indentation from paragraphs

\renewcommand{\labelenumi}{\alph{enumi}.} % Make numbering in the enumerate environment by letter rather than number (e.g. section 6)

%\usepackage{times} % Uncomment to use the Times New Roman font

%----------------------------------------------------------------------------------------
%	DOCUMENT INFORMATION
%----------------------------------------------------------------------------------------

\title{ COMP 3314 / CSIS 0314\\ Machine Learning \\ Assignment 4} % Title

\author{QIAN Xin, CHEN Xinxi} % Author name

\date{\today} % Date for the report

\begin{document}

\maketitle % Insert the title, author and date

\begin{center}
\begin{tabular}{l r}
Due Date: & May 5, 2015, 11:30 pm \\ % Date the experiment was performed
Instructor: & Jack Wang, Lu Shirui % Instructor/supervisor
\end{tabular}
\end{center}

% If you wish to include an abstract, uncomment the lines below
% \begin{abstract}
% Abstract text
% \end{abstract}

%----------------------------------------------------------------------------------------
%	SECTION 1
%----------------------------------------------------------------------------------------

\section{Introduction}

At a high-level, what are the approaches you considered and why?


\clearpage

%----------------------------------------------------------------------------------------
%	SECTION 2
%----------------------------------------------------------------------------------------

\section{Method}

Describe the approach you chose. For example, how did you preprocess the
data and what algorithms did you implement/use?

To use the libsvm library, the data format should be double. Therefore, we preprocess the data using 
I2 = im2double(I) converts the intensity image I to double precision, rescaling the data if necessary. I can be a grayscale intensity image, a truecolor image, or a binary image.
If the input image is of class double, then the output image is identical.

You may find it in ToDouble.

Many options in using LIBSVM

If n is small and m is intermediate, then use SVM with a Gaussian Kernel.




\clearpage

%----------------------------------------------------------------------------------------
%	SECTION 3
%----------------------------------------------------------------------------------------

\section{Results}

\clearpage

%----------------------------------------------------------------------------------------
%	SECTION 4
%----------------------------------------------------------------------------------------

\section{Conclusions}

Summarize what you’ve learned from the experiments and speculate on how you might improve the performance.

What makes your result good, and why? and/or,
What makes your result unsatisfying, why, and how to improve?

\clearpage



%----------------------------------------------------------------------------------------
%	BIBLIOGRAPHY
%----------------------------------------------------------------------------------------
\section{Reference}
References to any publically available code you used or writings you drew ideas from.
\clearpage

%----------------------------------------------------------------------------------------


\end{document}